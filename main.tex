\documentclass[12pt, twoside]{article}

\usepackage{amsmath}
\usepackage{amssymb}
\usepackage{fancyhdr}
\usepackage{float}
\usepackage[T1]{fontenc}
\usepackage{graphicx}
\usepackage[utf8]{inputenc}
\usepackage{mathtools}
\usepackage{txfonts}
\usepackage[normalem]{ulem}
\usepackage{wasysym}
%\usepackage[backend=biber, sorting=none]{biblatex}
\usepackage[style=ieee, backend=biber, sorting=none]{biblatex}
\addbibresource{bibliography.bib}
\usepackage[paperheight=29.69cm,paperwidth=21.01cm,left=2.54cm,right=2.54cm,top=2.54cm,bottom=2.54cm]{geometry}
\usepackage{pgfgantt}
\usepackage{titling}
\usepackage{titlesec}
\usepackage{caption} % Per personalitzar els peus de figura
\usepackage{pgfplots} % For graphs
\usepackage{lipsum} % For lorem ipsum placeholder text

\pgfplotsset{compat=1.18} % Ensure compatibility

% Avoid indentation for all paragraphs
\setlength{\parindent}{0pt}

\setlength{\parskip}{0.5em}

% Define the custom header command in the preamble
\newcommand{\header}[1]{%
    \vspace{1\baselineskip}
    \noindent\textbf{#1}%
}

\newcommand{\br}[1]{%
    \vspace{1\baselineskip}
}

\title{Titol}
\author{Autor}

% Configuració de capçaleres amb fancyhdr
\pagestyle{fancy}
\fancyhf{} % Esborra capçaleres i peus predeterminats

% Capçalera pàgines senars
\fancyhead[LO]{\makebox[\textwidth][l]{\textit{\fontsize{10}{12}\selectfont\nouppercase{\leftmark}}\hfill\textit{\fontsize{10}{12}\selectfont\thepage}}}

% Capçalera pàgines parells
\fancyhead[RE]{\makebox[\textwidth][l]{\textit{\fontsize{10}{12}\selectfont\thepage}\hfill\textit{\fontsize{10}{12}\selectfont Titol – Avantprojecte}}}

% Assegurar-se que totes les seccions comencen en pàgina senar
\newcommand{\sectionbreak}{\cleardoublepage} % Per forçar una nova pàgina senar

% Configuració de numeració per a seccions com a capítols
\renewcommand{\sectionmark}[1]{\markboth{#1}{}}

% Ajust de la capçalera
\renewcommand{\headrulewidth}{0pt} % Elimina la línia sota la capçalera

% Configuració del format del peu de figura
\captionsetup{
    format=plain,
    labelsep=period, % Punt després del número de la figura
}

% Redefine \thesubsubsection to avoid showing '0' in numbering
\makeatletter
\renewcommand{\thesubsubsection}{%
  \ifnum\value{subsubsection}=0
    \thesubsection % Use only subsection if subsubsection is 0
  \else
    \thesubsection.\arabic{subsubsection} % Normal numbering if subsubsection exists
  \fi
}
\makeatother

% Configure the numbering of figures and tables
\renewcommand{\thefigure}{\thesubsubsection.\arabic{figure}}
\renewcommand{\thetable}{\thesubsubsection.\arabic{table}}
\counterwithin{figure}{subsubsection}
\counterwithin{table}{subsubsection}

\usepackage{tocloft}
% Adjust spacing for List of Figures and Tables
\setlength{\cftfignumwidth}{3.5em} % Adjust width for figure numbers
\setlength{\cfttabnumwidth}{3.5em} % Adjust width for table numbers


\begin{document}


\thispagestyle{empty}
\begin{center}
\resizebox{0.8\textwidth}{!}{\includegraphics{logo_tcm_upc_docs.png}}\\

\vspace{2cm}
\large{Degree in Computer Engineering for Management and Information Systems}

\par
\vspace{4cm}

\huge{\thetitle}

\vspace{3cm}
\huge{Avantprojecte}

\end{center}

\vspace{3.5cm}
\begin{flushright}
    
\large{\theauthor}

\large{Tutor: Tutor}

\large{2024-25}

\end{flushright}

\begin{center}

\resizebox{0.5\textwidth}{!}{\includegraphics{logo_tcm.png}}\\

\end{center}

\newpage

%%%%%%%%%%%%  This Produces Abstract  %%%%%%%%%%%%
%\pagenumbering{gobble} % Amaga la numeració de pàgina
%\section*{Abstract}
%Aquest és el contingut de l'abstract o resum.

%\section*{Resum}
%Aquest és el contingut de l'abstract o resum.

%\section*{Resumen}
%Aquest és el contingut de l'abstract o resum.

% Comença la numeració des de l'índex
\clearpage
\thispagestyle{empty}

%%%%%%%%%%%%  This Produces Table Of Contents  %%%%%%%%%%%%
\pagenumbering{roman} % Reinicia el format de números de pàgina
\setcounter{page}{1}
\tableofcontents
\clearpage

\addcontentsline{toc}{section}{List of Figures}
\listoffigures
\clearpage

\addcontentsline{toc}{section}{List of Tables}
\listoftables
\clearpage

\section{Introduction}
\lipsum[1] % Placeholder text

\subsection{Background}
\lipsum[2] % Placeholder text

\subsubsection{Related Work}
\lipsum[3] % Placeholder text

% Example figure
\begin{figure}[H]
    \centering
    \includegraphics[width=0.5\textwidth]{example-image} % Replace with your image path
    \caption{Example figure showing a placeholder image.}
    \label{fig:example}
\end{figure}

% Example table
\begin{table}[H]
    \centering
    \begin{tabular}{|c|c|c|}
        \hline
        Column 1 & Column 2 & Column 3 \\ \hline
        Data 1   & Data 2   & Data 3   \\ \hline
        Data 4   & Data 5   & Data 6   \\ \hline
    \end{tabular}
    \caption{Example table with placeholder data.}
    \label{tab:example}
\end{table}

\subsection{Another Section}
\lipsum[4-5] % More placeholder text

\subsubsection{Figures and Tables}
Referencing Figure~\ref{fig:example} and Table~\ref{tab:example} is straightforward.

\header{Header}

Like a subsubsubsection that does not show up in the table of contents.

\begin{figure}[H]
    \centering
    \begin{tikzpicture}
        \begin{axis}[
            axis lines=middle,
            xlabel=$x$, ylabel=$f(x)$,
            xmin=-2, xmax=2, ymin=-1, ymax=2,
            samples=200,
            domain=-2:2,
            grid=both,
            grid style={dotted, gray},
            xtick={-2,-1,0,1,2},
            ytick={-1,0,1},
            enlargelimits,
            width=10cm, height=7cm
        ]
            \addplot[thick, blue, domain=-2:2] {max(0,x)};
        \end{axis}
    \end{tikzpicture}
    \caption{Graph of the ReLU function.}
    \label{fig:relu_graph}
\end{figure}

\section{Conclusion}
\lipsum[6] % Placeholder text
\cite{vaswani2017}

\newpage

\medskip

\addcontentsline{toc}{section}{References}
\printbibliography

\end{document}